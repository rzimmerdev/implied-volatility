\documentclass{article}

\usepackage{arxiv}

\usepackage[utf8]{inputenc} % allow utf-8 input
\usepackage[T1]{fontenc}    % use 8-bit T1 fonts
\usepackage{hyperref}       % hyperlinks
\usepackage{url}            % simple URL typesetting
\usepackage{booktabs}       % professional-quality tables
\usepackage{amsfonts}       % blackboard math symbols
\usepackage{nicefrac}       % compact symbols for 1/2, etc.
\usepackage{microtype}      % microtypography
\usepackage{graphicx}
\usepackage{natbib}
\usepackage{doi}
\usepackage[portuguese]{babel}

\title{Transformers para Geração de Superfícies Suaves de Volatilidade Implícita}

\author{ 
	\href{https://orcid.org/0009-0008-6064-9895}{
		\includegraphics[scale=0.06]{resources/orcid.pdf}
		\hspace{1mm}Rafael Zimmer
	}\\
	Institute of Mathematical Science and Computing\\
	University of São Paulo - Brazil \\
	\texttt{rafael.zimmer@usp.br} \\
	\And
	\href{https://orcid.org/0000-0000-0000-0000}{
		\includegraphics[scale=0.06]{resources/orcid.pdf}
		\hspace{1mm}Adalton Sena Filho
	}\\
	Institute of Mathematical Science and Computing\\
	University of São Paulo - Brazil \\
	\texttt{adaltonsena@usp.br} \\
	\And
	\href{https://orcid.org/0000-0000-0000-0000}{
		\includegraphics[scale=0.06]{resources/orcid.pdf}
		\hspace{1mm}Eduardo Brito
	}\\
	Institute of Mathematical Science and Computing\\
	University of São Paulo - Brazil \\
	\texttt{eduardobritocusp.br} \\
}

%\date{}

\renewcommand{\headeright}{}
\renewcommand{\undertitle}{}
\renewcommand{\shorttitle}{Transformers para Geração de Superfícies Suaves de Volatilidade Implícita}

\hypersetup{
	pdftitle={Transformers para Geração de Superfícies Suaves de Volatilidade Implícita},
	pdfsubject={q-bio.NC, q-bio.QM},
	pdfauthor={Zimmer, R., Filho, A., Brito, E.},
	pdfkeywords={First keyword, Second keyword, More},
}

% Redefine abstract name to Resumo
\addto\captionsportuguese{\renewcommand{\abstractname}{Resumo}}

\begin{document}
	\selectlanguage{portuguese}
	
	\maketitle
	
	\begin{abstract}
		Neste trabalho implementamos uma rede neural com arquitetura baseada em \textit{Transformers} para otimização de valores $P, Q, R$ para diversos modelos \textit{SABR} paramétricos (\textit{parametric stochastic $\alpha, \beta, \rho$}) para geração de superfícies de volatilidade implicita no contexto de opções financeiras. A motivação por trás do trabalho é permitir a interpolação e extrapolação suave da superfície, ou seja, tornando-a derivável em todos os pontos. O desenvolvimento da arquitetura tem como principal contribuição solucionar o problema na escolha dos candidatos $\alpha, \rho, volvol$ cujos parâmetros $P, Q, R$ otimizados gerem a superfície com menor oportunidade de arbitragem. Por fim, utilizaremos os valores históricos observados de $\alpha, \rho, volvol$ e a capacidade inata das arquiteturas de transformers para processar observações de sequências com tamanhos não fixos, ou seja, para volatilidades de contratos com data de vencimento variadas e não fixas e gerar as superfícies, comparando os resultados com métodos de interpolação clássicos (linear e \textit{splines}).
	\end{abstract}
	
	\keywords{First keyword \and Second keyword \and More}
	
	\section{Introdução}
\subsection{Uma breve introdução à volatilidade implícita e a reconstrução de superfícies}
A superfície de volatilidade implícita desempenha um papel crítico no contexto de contratos de derivativos financeiros, especificamente contratos de opções \citep{Ludkovski2023}. Uma opção é um tipo derivativo, ou seja, um ativo negociado em relação a outro ativo, em que o comprador do contrato adquire a opção de realizar ou não a compra ou venda de um determinado ativo para com o vendedor do contrato na data futura. O cálculo de requisitos de margem até a precificação destes derivativos dependem de uma gama de valores, entre eles o tempo até o vencimento do contrato, o preço do ativo a ser negociado, e o preço do ativo a ser negociado no futuro \citep{Black1973}. Há também uma variável adicional para a precificação do contrato, chamada de volatilidade implicita (\textit{IVOL}). Tal valor refere-se à volatilidade considerada, ou esperada, para o preço do ativo no futuro, e é essencial para o uso na equação de Black-Scholes Merton, para precificação de contratos de opções \citep{Merton1973}. 

Ter acesso aos valores corretos de volatilidade implicita permite a investidores realizarem vendas pelos preços corretos e condizentes com os valores reais dos contratos no mercado, contudo, nem sempre é possível obter valores da \textit{IVOL} que reflitam totalmente as características das ofertas existentes. Garantir uma forma de obter esse valor com confiança e com os dados disponíveis no mercado é uma tarefa essencial para analistas quantitativos \citep{Kim2006}. Estratégias tradicionais para cálculo diário destes valores são fundamentadas predominantemente em algoritmos de interpolação \citep{Homescu2011} ou modelos de equações diferenciais estocásticas \citep{Jordan2011}, ambos buscando aproximar para cada par de preço de vencimento (também chamada de maturidade) e preço do ativo no futuro (também chamada de \textit{strike}) um valor correto para a volatilidade implicita. Essa abordagem permite visualizar os resultados do modelo no formato de uma superfície, onde os eixos \textbf{XY} são os pares de maturidade e \textit{strikes} e o eixo \textbf{Z} é a \textit{IVOL}, também chamada de superfície de volatilidade implicita (SVI). 

Garantir previsões precisas da SVI é fundamental, pois até pequenos erros podem levar a precificações incorretas e consequentemente perda de lucro ou até retorno negativo (chamados nesse contexto de oportunidades de arbitragem) \citep{Hagan2014}. Modelos tradicionais como interpolações lineares, polinomiais e \textit{splines} geram oportunidades de arbitragem na SVI, e muitas vezes lutam para capturar efetivamente a dinâmica do mercado, além de carecer flexibilidade para extrapolação \citep{Hagan2014}.

\subsection{Contribuições do Trabalho}

Neste artigo, propomos uma abordagem para enfrentar os desafios de interpolação e extrapolação da SVI, reduzindo oportunidades de arbitragem. O método escolhido usa uma abordagem baseada em soluções de equações diferenciais estocásticas, em específico o modelo \textit{Stochastic Alpha, Beta, Rho} (\textit{SABR}). Serão utilizados os valores históricos das volatilidades implicitas observadas junto de um otimizador não-linear para obter os parâmetros que minimizem o erro da superfície.

A otimização dos parâmetros do modelo \textit{SABR} é comumente realizada utilizando apenas um subconjunto dos pontos observados diariamente \citep{Kim2021}, devido ao fato de utilizar todos os pontos insere diversos \textit{outliers} que não estão de acordo com os preços reais do derivativo, inserindo no modelo otimizado oportunidades de arbitragem. Para solucionar essa dificuldade, comumente os parâmetros são filtrados manualmente de acordo com critérios de analistas especialistas \citep{Hagan2002}, e um subconjunto dos pontos candidatos é utilizado para a otimização dos parâmetros. O uso de modelos capazes de receber como entrada nuvems de pontos (\textit{point clouds} em inglês) permite filtrar pontos candidatos para serem utilizados pelos otimizadores não-lineares de forma muito mais rápida e automatizada. A contribuição principal desta pesquisa será no uso de arquiteturas de \textit{transformers}, que ao contrário das redes neurais convencionais, aceitam quantidades variáveis de parâmetros para um determinado dia e diferentemente de redes neurais recorrentes é mais apropriado para lidar com nuvens de pontos não ordenadas por tempo \citep{Kim2024}. Os processos de otimização e filtragem será abordado mais a fundo na seção de desenvolvimento.

\subsection{A arquitetura do \textit{Transformer}}
O modelo \textit{SABR} é um modelo de volatilidade estocástico utilizado frequentemente para obter superficies suaves, ou seja, contínuas e infinitamente deriváveis em todos os pontos \citep{Hagan2015}. No contexto do modelo SABR, à cada um dos parâmetros candidatos será atribuido uma pontuação chamada de valor \textit{SS}, que corresponde à aptidão dos candidatos em questão de serem resumidos ou representados por um subconjunto de pontos, ou seja, quão possível é reduzir a quantidade de candidatos sem alterar o formato da superfície. Em suma, esse valor de pontuação é normalizado, e utilizado para escolher apenas os pontos com maior pontuação, retirando no processo \textit{outliers} que possam introduzir arbitragem no processo de otimização dos modelos \textit{SABR}.

Usualmente, a escolha dos subconjuntos e o processo de pontuação é feito manualmente por analistas, como mencionado anteriormente, e a próxima seção focará nos detalhes abrangendo o modelo \textit{SABR} e como utilizamos uma rede neural com atenção, ou seja, uma arquitetura de \textit{Transformer}, especificamente a camada de \textit{\textbf{encoder}} para a atribuição das pontuações de forma automática.

	
	\section{Problem description and Motivation}
	
	\section{Metodologia e Resultados}

\subsection{Coleta de Dados}

Os dados utilizados neste trabalho foram obtidos de dados históricos de opções sobre o ativo $SPY$, da bolsa estadunidense, de 2021 à 2022, disponível no \href{https://www.kaggle.com/datasets/shawlu/option-spy-dataset-combinedcsv}{Kaggle}. Este conjunto de dados contém diversas colunas sobre opções negociadas, indexadas pela data, incluindo preços de exercício, preços futuros, maturidades, volatilidades implícitas, entre outros \citep{}. O arquivo pode ser e foi baixado e armazenado localmente em um diretório específico para processamento.

\subsection{Descrição do Conjunto de Dados}

O conjunto de dados contém as seguintes colunas relevantes para nosso estudo, além de diversas outras:
\begin{itemize}
	\item \textbf{underlying}: Preço do ativo subjacente.
	\item \textbf{strike}: Preço de exercício da opção.
	\item \textbf{daysToExpiration}: Dias restantes até a expiração da opção.
	\item \textbf{iv}: Volatilidade implícita.
	\item \textbf{dt}: Data do registro.
\end{itemize}

Além disso, adicionamos duas colunas calculadas:
\begin{itemize}
	\item \textbf{maturity}: Calculada dividindo \textit{daysToExpiration} por 360 (dias úteis no ano) para normalizar o tempo de maturidade.
	\item \textbf{r}: Taxa livre de risco obtida do Federal Reserve Economic Data (FRED) para o período correspondente.
	\item \textbf{d}: Taxa de dividendos paga trimestralmente pelo SPY.
\end{itemize}

\subsection{Pré-processamento dos Dados}

O pré-processamento dos dados encorreu em várias etapas essenciais, predominantemente realizadas para permitir o uso dos dados diretamente na arquitetura do transformer e subsequente uso para gerar a superfície para os dados históricos. O transformer tem como entrada 4 dimensões, sendo elas: 
\begin{enumerate}
\item O tempo até a maturidade; 
\item O valor do parâmetro ($\alpha, \rho$, ou $\nu$); 
\item $\mathbf{1}$ se o parâmetro foi calculado para o dia atual, $\mathbf{0}$ se foi calculado no dia anterior por não existir no dia atual; 
\item Parâmetros calculados utilizando as pontuações do dia anterior.
\end{enumerate}

A variável target são os scores gerados utilizando-se todos os pontos candidatos para o dia atual, e a função de erro do modelo utiliza uma porcentagem \textbf{p} de pontos e é treinado para minimizar a função de erro quadrática das pontuações.

\subsection{Métricas de Erro e Comparação das Superfícies}

Utilizamos como baseline uma interpolação linear dos pontos observados diariamente, e comparamos com a performance do modelo em relação à pontos internos à uma malha (ou \textit{grid} em inglês) gerada para pontos dentro do intervalo diário, tanto para valores de maturidade como de \textit{strike}.

O nosso modelo transformer foi treinado em cima do conjunto de dados, de aproximadamente 70 observações, cada uma com uma sequência de em média 700 pontos. Os modelos foram treinados por aproximadamente 700 épocas.

\begin{figure}
	\begin{center}
		\includegraphics[width=8cm]{resources/losses.png}
		\caption{Função de perda histórica, de uma sessão de treinamento do Transformer}
	\end{center}
\end{figure}
	
\begin{figure}
	\begin{center}
		\includegraphics[width=8cm]{resources/pred_surface.png}
		\caption{Superfície gerada utilizando um modelo SABR com parâmetros otimizados manualmente. Note que a não-remoção de \textit{outliers} gerou um pico na superfície}
	\end{center}
\end{figure}

\begin{figure}
	\begin{center}
		\includegraphics[width=16cm]{resources/sst.png}
		\caption{Comparação da superfície real com a gerada após treinamento do Transformer. Na esquerda, a superfície do modelo SST, na direita, a superfície interpolada linearmente}
	\end{center}
\end{figure}
	

	
	\section{Conclusão}
\subsection{Resumo}

Este trabalho apresentou uma abordagem para ajustar os parâmetros do modelo SABR utilizando uma rede neural baseada na arquitetura de transformers (especificamente o elemento do \textit{encoder}). A combinação do modelo SABR com a arquitetura transformer foi utilizada como forma de replicar os padrões da superfície de volatilidade implicita dia à dia, gerando a superfície suave e sem oportunidade de arbitragem para os contratos existentes observados no dia. O fato da escolha e pontuação dos candidatos ser possível de ser realizada com um modelo automático, em vez de manualmente, é uma solução extremamente eficiente, principalmente se considerarmos a possibilidade de utilizar esses modelos na GPU, o que permite a reconstrução da superfície em instantes, permitindo investidores identificar oportunidades de negociação em apenas poucos instantes, em vez de em uma escala diária.

\subsection{Implicações e Trabalhos Futuros}

O principal foco do trabalho é realmente a abordagem utilizando o transformer para substituir a escolha manual das pontuações para os candidatos, abrindo caminho para eventuais novas pesquisas e estratégias de High-Frequency Trading (ou \textit{HFT}, abreviação em inglês). Pesquisas futuras podem explorar a integração de fatores de mercado adicionais e aprimorar a arquitetura que escolhemos, possivelmente aumentando a quantidade de blocos ou cabeças de atenção. Além disso, é importante considerar a escolha do otimizador dos parâmetros \textbf{P, Q, R}, pois o tempo para realizar a etapa de preprocessamento foi um grande limitante no decorrer do trabalho. Estender o modelo para outros instrumentos financeiros também é uma possibilidade, pois permite o estudo sobre a adaptação do modelo a diferentes regimes de mercado e condições econômicas (tanto micro- como macro-econômicas) variáveis podem revelar novas oportunidades de aplicação e desenvolvimento do método proposto.



	
	\bibliographystyle{unsrtnat}
	\bibliography{bibliography}
\end{document}
