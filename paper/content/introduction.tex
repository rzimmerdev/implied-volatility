\section{Introduction}
\subsection{A short introduction on the Implied Volatility Surface}
The implied volatility surface (IVS) plays a critical role in various financial applications, ranging from margin requirements calculation to pricing exotic derivatives. Ensuring accurate predictions of the IVS is paramount, as even small errors can lead to significant financial implications. Traditional models for the IVS often struggle to effectively capture market dynamics and lack the flexibility to value instruments without quoted prices.

In this paper, we propose a novel approach to address the challenges of interpolating and extrapolating the IVS by leveraging implicit neural layers. Our method integrates insights from both machine learning (ML) and mathematical finance, offering a promising solution to this complex financial problem. Unlike conventional neural networks (NNs), which directly model the IVS, our approach utilizes implicit layers within fixed point iteration to enhance model interpretability and reduce computational complexity.

The motivation for our research stems from the need to improve existing IVS models while maintaining a balance between flexibility and computational efficiency. Our method enables practitioners to augment standard models with our approach, providing a robust framework for IVS prediction. The applicability of our method extends beyond options to various other financial problems, showcasing its versatility and potential impact across the financial industry.

\subsection{Background and context}
In finance, options represent financial contracts granting the holder the right to buy (call option) or sell (put option) an asset at a predetermined price and date. Option pricing traditionally relies on models such as the Black-Scholes formula, which, while widely used, comes with inherent limitations and assumptions that may not fully capture market realities. The challenge lies in constructing a continuous representation of volatility, known as the IVS, from a finite set of observed option prices, while ensuring the absence of arbitrage opportunities.

Existing literature has explored the application of NNs for implied volatility smoothing, capitalizing on their universal approximation properties. However, conventional NN approaches lack interpretability and may struggle with incorporating domain-specific knowledge essential for financial modeling. Our work aims to bridge this gap by integrating domain expertise from mathematical finance with the flexibility of NNs, offering a consistent and well-performing hybrid model for IVS prediction.

\subsection{Contributions and significance of the proposed solution}
We present a methodology that combines implicit neural layers with prior models to correct, interpolate, and extrapolate the IVS in an arbitrage-free manner. Our approach leverages soft constraints derived from mathematical finance to guide model training, ensuring realistic and flexible predictions. Through extensive numerical experiments and empirical analysis, we demonstrate the effectiveness of our method in capturing market dynamics and enhancing IVS prediction accuracy.

The significance of our work lies in its ability to provide a reliable and computationally efficient framework for IVS prediction, bridging the gap between traditional finance challenges and recent advancements in deep learning. By introducing interpretability through implicit layers, our approach offers a promising avenue for enhancing financial modeling practices. Our research contributes to the ongoing exploration of NN applications in finance and underscores the importance of integrating domain expertise with advanced machine learning techniques.