\section{Introdução}
\subsection{Uma breve introdução à volatilidade implícita e a reconstrução de superfícies}
A superfície de volatilidade implícita desempenha um papel crítico no contexto de contratos de derivativos financeiros, especificamente contratos de opções \citep{Ludkovski2023}. Uma opção é um tipo derivativo, ou seja, um ativo negociado em relação a outro ativo, em que o comprador do contrato adquire a opção de realizar ou não a compra ou venda de um determinado ativo para com o vendedor do contrato na data futura. O cálculo de requisitos de margem até a precificação destes derivativos dependem de uma gama de valores, entre eles o tempo até o vencimento do contrato, o preço do ativo a ser negociado, e o preço do ativo a ser negociado no futuro \citep{Black1973}. Há também uma variável adicional para a precificação do contrato, chamada de volatilidade implicita (\textit{IVOL}). Tal valor refere-se à volatilidade considerada, ou esperada, para o preço do ativo no futuro, e é essencial para o uso na equação de Black-Scholes Merton, para precificação de contratos de opções \citep{Merton1973}. 

Ter acesso aos valores corretos de volatilidade implicita permite a investidores realizarem vendas pelos preços corretos e condizentes com os valores reais dos contratos no mercado, contudo, nem sempre é possível obter valores da \textit{IVOL} que reflitam totalmente as características das ofertas existentes. Garantir uma forma de obter esse valor com confiança e com os dados disponíveis no mercado é uma tarefa essencial para analistas quantitativos \citep{Kim2006}. Estratégias tradicionais para cálculo diário destes valores são fundamentadas predominantemente em algoritmos de interpolação \citep{Homescu2011} ou modelos de equações diferenciais estocásticas \citep{Jordan2011}, ambos buscando aproximar para cada par de preço de vencimento (também chamada de maturidade) e preço do ativo no futuro (também chamada de \textit{strike}) um valor correto para a volatilidade implicita. Essa abordagem permite visualizar os resultados do modelo no formato de uma superfície, onde os eixos \textbf{XY} são os pares de maturidade e \textit{strikes} e o eixo \textbf{Z} é a \textit{IVOL}, também chamada de superfície de volatilidade implicita (SVI). 

Garantir previsões precisas da SVI é fundamental, pois até pequenos erros podem levar a precificações incorretas e consequentemente perda de lucro ou até retorno negativo (chamados nesse contexto de oportunidades de arbitragem) \citep{Hagan2014}. Modelos tradicionais como interpolações lineares, polinomiais e \textit{splines} geram oportunidades de arbitragem na SVI, e muitas vezes lutam para capturar efetivamente a dinâmica do mercado, além de carecer flexibilidade para extrapolação \citep{Hagan2014}.

\subsection{Contribuições do Trabalho}

Neste artigo, propomos uma abordagem para enfrentar os desafios de interpolação e extrapolação da SVI, reduzindo oportunidades de arbitragem. O método escolhido usa uma abordagem baseada em soluções de equações diferenciais estocásticas, em específico o modelo \textit{Stochastic Alpha, Beta, Rho} (\textit{SABR}). Serão utilizados os valores históricos das volatilidades implicitas observadas junto de um otimizador não-linear para obter os parâmetros que minimizem o erro da superfície.

A otimização dos parâmetros do modelo \textit{SABR} é comumente realizada utilizando apenas um subconjunto dos pontos observados diariamente \citep{Kim2021}, devido ao fato de utilizar todos os pontos insere diversos \textit{outliers} que não estão de acordo com os preços reais do derivativo, inserindo no modelo otimizado oportunidades de arbitragem. Para solucionar essa dificuldade, comumente os parâmetros são filtrados manualmente de acordo com critérios de analistas especialistas \citep{Hagan2002}, e um subconjunto dos pontos candidatos é utilizado para a otimização dos parâmetros. O uso de modelos capazes de receber como entrada nuvems de pontos (\textit{point clouds} em inglês) permite filtrar pontos candidatos para serem utilizados pelos otimizadores não-lineares de forma muito mais rápida e automatizada. A contribuição principal desta pesquisa será no uso de arquiteturas de \textit{transformers}, que ao contrário das redes neurais convencionais, aceitam quantidades variáveis de parâmetros para um determinado dia e diferentemente de redes neurais recorrentes é mais apropriado para lidar com nuvens de pontos não ordenadas por tempo \citep{Kim2024}. Os processos de otimização e filtragem será abordado mais a fundo na seção de desenvolvimento.

\subsection{A arquitetura do \textit{Transformer}}
O modelo \textit{SABR} é um modelo de volatilidade estocástico utilizado frequentemente para obter superficies suaves, ou seja, contínuas e infinitamente deriváveis em todos os pontos \citep{Hagan2015}. No contexto do modelo SABR, à cada um dos parâmetros candidatos será atribuido uma pontuação chamada de valor \textit{SS}, que corresponde à aptidão dos candidatos em questão de serem resumidos ou representados por um subconjunto de pontos, ou seja, quão possível é reduzir a quantidade de candidatos sem alterar o formato da superfície. Em suma, esse valor de pontuação é normalizado, e utilizado para escolher apenas os pontos com maior pontuação, retirando no processo \textit{outliers} que possam introduzir arbitragem no processo de otimização dos modelos \textit{SABR}.

Usualmente, a escolha dos subconjuntos e o processo de pontuação é feito manualmente por analistas, como mencionado anteriormente, e a próxima seção focará nos detalhes abrangendo o modelo \textit{SABR} e como utilizamos uma rede neural com atenção, ou seja, uma arquitetura de \textit{Transformer}, especificamente a camada de \textit{\textbf{encoder}} para a atribuição das pontuações de forma automática.
