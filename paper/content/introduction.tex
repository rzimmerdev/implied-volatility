\section{Introdução}
\subsection{Uma breve introdução à volatilidade implícita e a reconstrução de superfícies}
A superfície de volatilidade implícita (SVI) desempenha um papel crítico em várias aplicações financeiras, desde o cálculo de requisitos de margem até a precificação de derivativos exóticos. Garantir previsões precisas da SVI é fundamental, pois até pequenos erros podem levar a implicações financeiras significativas. Modelos tradicionais para a SVI muitas vezes lutam para capturar efetivamente a dinâmica do mercado e carecem da flexibilidade para valorar instrumentos sem preços cotados.

Neste artigo, propomos uma abordagem para enfrentar os desafios de interpolação e extrapolação da SVI, reduzindo oportunidades de arbitragem. O método escolhido visa implementar arquiteturas de \textit{transformers}, que ao contrário das redes neurais convencionais, aceitam quantidades variáveis de parâmetros históricos para modelos de superfícies suaves. A motivação para a abordagem escolhida decorre da dificuldade na escolha dos candidatos para otimização do modelo \textit{Stochastic Alpha, Beta, Rho} (\textit{SABR}), que usualmente é feita manualmente por analistas. O uso de modelos capazes de receber como entrada pontos em nuvem (\textit{point clouds} em inglês, quando abordado no contexto de arquiteturas de Transformers) permite filtrar pontos candidatos para serem utilizados pelos otimizadores não-lineares. O processo de otimização e filtragem será abordado mais a fundo na seção de desenvolvimento.

\subsection{Contexto e antecedentes}
Em finanças, opções representam contratos financeiros que concedem ao titular o direito de comprar (opção de compra) ou vender (opção de venda) um ativo a um preço e maturidade (ou data) predeterminados. A precificação de opções tradicionalmente depende de modelos como a fórmula de Black-Scholes, que, embora amplamente utilizada, possui limitações e pressupostos inerentes que podem não capturar totalmente as realidades do mercado. O desafio está em construir um mapeamento contínuo da volatilidade, conhecida como SVI, a partir de um conjunto finito de preços de opções observados, garantindo a ausência de oportunidades de arbitragem.

A literatura existente explora o uso de Redes Neurais (\textit{Neural Networks} ou NN em inglês) para suavização da volatilidade implícita, no entanto, abordagens convencionais de NNs aproximam os valores da volatilidade diretamente, o que gera oportunidades de arbitragem por gerar superfícies cujos pontos observáveis não condizem com os existentes no mercado.

O modelo SABR é uma alternativa para esse problema, sendo possível aproximar os parâmetros $\alpha, \beta, \rho$ em vez da volatilidade em si. Para isso, contudo é necessário ter-se um conjunto de pontos observados para realizar uma otimização desses valores em função da volatilidade observada. Essa questão gera diversos problemas, devido a necessidade de uma escolha boa (que minimize o erro e desvio padrão para com a volatilidade observada) de observações para ser utilizada. Essa dificuldade será o foco no desenvolvimento do modelo deste trabalho, e será discutida mais a fundo na seção de desenvolvimento.

\subsection{Contribuições}

\begin{enumerate}
	\item \textbf{Desenvolvimento de um modelo SABR com cálculo vetorizado de volatilidade implícita}: A implementação proposta do modelo SABR calcula eficientemente as volatilidades implícitas usando operações vetorizadas.
	\item \textbf{Integração com uma abordagem de ajuste de parâmetros baseada em transformer}: O modelo transformer é empregado para ajustar os parâmetros do SABR, capturando padrões complexos nos dados históricos de opções.
	\item \textbf{Geração de superfícies de volatilidade suaves}: O modelo combinado gera superfícies de volatilidade diárias para o conjunto de dados de opções do ativo americano $SPY$, replicando a dinâmica das superfícies históricas.
\end{enumerate}