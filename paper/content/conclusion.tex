\section{Conclusão}
\subsection{Resumo}

Este trabalho apresentou uma abordagem para ajustar os parâmetros do modelo SABR utilizando uma rede neural baseada na arquitetura de transformers (especificamente o elemento do \textit{encoder}). A combinação do modelo SABR com a arquitetura transformer foi utilizada para replicar padrões da superfície de volatilidade implicita histórica, resultando reconstrução suave e sem oportunidade de arbitragem para os preços diários observados. A utilização de operações vetorizadas no cálculo de volatilidades implícitas e a integração com um método de ajuste de parâmetros baseado em transformer demonstraram melhorias significativas na modelagem da volatilidade diária, relativo à eficiência em escolher os candidatos para otimização manualmente.

\subsection{Implicações e Trabalhos Futuros}

A capacidade de gerar superfícies de volatilidade diárias suaves apoia diversas aplicações, incluindo a precificação de opções, a gestão de riscos e o trading de derivativos. Além disso, a abordagem apresentada destaca o potencial dos transformers na modelagem de dados financeiros complexos, abrindo caminho para novas pesquisas e inovações no campo da engenharia financeira.

Pesquisas futuras podem explorar a integração de fatores de mercado adicionais e arquiteturas de redes neurais alternativas para aprimorar ainda mais a precisão das previsões de volatilidade. Estender o modelo para outros instrumentos financeiros e mercados proporcionará uma validação mais ampla de sua eficácia. Além disso, investigações sobre a adaptação do modelo a diferentes regimes de mercado e condições econômicas variáveis podem revelar novas oportunidades de aplicação e desenvolvimento do método proposto.
