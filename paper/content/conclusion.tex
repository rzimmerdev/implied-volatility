\section{Conclusão}
\subsection{Resumo}

Este trabalho apresentou uma abordagem para ajustar os parâmetros do modelo SABR utilizando uma rede neural baseada na arquitetura de transformers (especificamente o elemento do \textit{encoder}). A combinação do modelo SABR com a arquitetura transformer foi utilizada como forma de replicar os padrões da superfície de volatilidade implicita dia à dia, gerando a superfície suave e sem oportunidade de arbitragem para os contratos existentes observados no dia. O fato da escolha e pontuação dos candidatos ser possível de ser realizada com um modelo automático, em vez de manualmente, é uma solução extremamente eficiente, principalmente se considerarmos a possibilidade de utilizar esses modelos na GPU, o que permite a reconstrução da superfície em instantes, permitindo investidores identificar oportunidades de negociação em apenas poucos instantes, em vez de em uma escala diária.

\subsection{Implicações e Trabalhos Futuros}

O principal foco do trabalho é realmente a abordagem utilizando o transformer para substituir a escolha manual das pontuações para os candidatos, abrindo caminho para eventuais novas pesquisas e estratégias de High-Frequency Trading (ou \textit{HFT}, abreviação em inglês). Pesquisas futuras podem explorar a integração de fatores de mercado adicionais e aprimorar a arquitetura que escolhemos, possivelmente aumentando a quantidade de blocos ou cabeças de atenção. Além disso, é importante considerar a escolha do otimizador dos parâmetros \textbf{P, Q, R}, pois o tempo para realizar a etapa de preprocessamento foi um grande limitante no decorrer do trabalho. Estender o modelo para outros instrumentos financeiros também é uma possibilidade, pois permite o estudo sobre a adaptação do modelo a diferentes regimes de mercado e condições econômicas (tanto micro- como macro-econômicas) variáveis podem revelar novas oportunidades de aplicação e desenvolvimento do método proposto.


